% Options for packages loaded elsewhere
\PassOptionsToPackage{unicode}{hyperref}
\PassOptionsToPackage{hyphens}{url}
%
\documentclass[
]{article}
\author{}
\date{\vspace{-2.5em}}

\usepackage{amsmath,amssymb}
\usepackage{lmodern}
\usepackage{iftex}
\ifPDFTeX
  \usepackage[T1]{fontenc}
  \usepackage[utf8]{inputenc}
  \usepackage{textcomp} % provide euro and other symbols
\else % if luatex or xetex
  \usepackage{unicode-math}
  \defaultfontfeatures{Scale=MatchLowercase}
  \defaultfontfeatures[\rmfamily]{Ligatures=TeX,Scale=1}
\fi
% Use upquote if available, for straight quotes in verbatim environments
\IfFileExists{upquote.sty}{\usepackage{upquote}}{}
\IfFileExists{microtype.sty}{% use microtype if available
  \usepackage[]{microtype}
  \UseMicrotypeSet[protrusion]{basicmath} % disable protrusion for tt fonts
}{}
\makeatletter
\@ifundefined{KOMAClassName}{% if non-KOMA class
  \IfFileExists{parskip.sty}{%
    \usepackage{parskip}
  }{% else
    \setlength{\parindent}{0pt}
    \setlength{\parskip}{6pt plus 2pt minus 1pt}}
}{% if KOMA class
  \KOMAoptions{parskip=half}}
\makeatother
\usepackage{xcolor}
\IfFileExists{xurl.sty}{\usepackage{xurl}}{} % add URL line breaks if available
\IfFileExists{bookmark.sty}{\usepackage{bookmark}}{\usepackage{hyperref}}
\hypersetup{
  hidelinks,
  pdfcreator={LaTeX via pandoc}}
\urlstyle{same} % disable monospaced font for URLs
\usepackage[margin=1in]{geometry}
\usepackage{color}
\usepackage{fancyvrb}
\newcommand{\VerbBar}{|}
\newcommand{\VERB}{\Verb[commandchars=\\\{\}]}
\DefineVerbatimEnvironment{Highlighting}{Verbatim}{commandchars=\\\{\}}
% Add ',fontsize=\small' for more characters per line
\usepackage{framed}
\definecolor{shadecolor}{RGB}{248,248,248}
\newenvironment{Shaded}{\begin{snugshade}}{\end{snugshade}}
\newcommand{\AlertTok}[1]{\textcolor[rgb]{0.94,0.16,0.16}{#1}}
\newcommand{\AnnotationTok}[1]{\textcolor[rgb]{0.56,0.35,0.01}{\textbf{\textit{#1}}}}
\newcommand{\AttributeTok}[1]{\textcolor[rgb]{0.77,0.63,0.00}{#1}}
\newcommand{\BaseNTok}[1]{\textcolor[rgb]{0.00,0.00,0.81}{#1}}
\newcommand{\BuiltInTok}[1]{#1}
\newcommand{\CharTok}[1]{\textcolor[rgb]{0.31,0.60,0.02}{#1}}
\newcommand{\CommentTok}[1]{\textcolor[rgb]{0.56,0.35,0.01}{\textit{#1}}}
\newcommand{\CommentVarTok}[1]{\textcolor[rgb]{0.56,0.35,0.01}{\textbf{\textit{#1}}}}
\newcommand{\ConstantTok}[1]{\textcolor[rgb]{0.00,0.00,0.00}{#1}}
\newcommand{\ControlFlowTok}[1]{\textcolor[rgb]{0.13,0.29,0.53}{\textbf{#1}}}
\newcommand{\DataTypeTok}[1]{\textcolor[rgb]{0.13,0.29,0.53}{#1}}
\newcommand{\DecValTok}[1]{\textcolor[rgb]{0.00,0.00,0.81}{#1}}
\newcommand{\DocumentationTok}[1]{\textcolor[rgb]{0.56,0.35,0.01}{\textbf{\textit{#1}}}}
\newcommand{\ErrorTok}[1]{\textcolor[rgb]{0.64,0.00,0.00}{\textbf{#1}}}
\newcommand{\ExtensionTok}[1]{#1}
\newcommand{\FloatTok}[1]{\textcolor[rgb]{0.00,0.00,0.81}{#1}}
\newcommand{\FunctionTok}[1]{\textcolor[rgb]{0.00,0.00,0.00}{#1}}
\newcommand{\ImportTok}[1]{#1}
\newcommand{\InformationTok}[1]{\textcolor[rgb]{0.56,0.35,0.01}{\textbf{\textit{#1}}}}
\newcommand{\KeywordTok}[1]{\textcolor[rgb]{0.13,0.29,0.53}{\textbf{#1}}}
\newcommand{\NormalTok}[1]{#1}
\newcommand{\OperatorTok}[1]{\textcolor[rgb]{0.81,0.36,0.00}{\textbf{#1}}}
\newcommand{\OtherTok}[1]{\textcolor[rgb]{0.56,0.35,0.01}{#1}}
\newcommand{\PreprocessorTok}[1]{\textcolor[rgb]{0.56,0.35,0.01}{\textit{#1}}}
\newcommand{\RegionMarkerTok}[1]{#1}
\newcommand{\SpecialCharTok}[1]{\textcolor[rgb]{0.00,0.00,0.00}{#1}}
\newcommand{\SpecialStringTok}[1]{\textcolor[rgb]{0.31,0.60,0.02}{#1}}
\newcommand{\StringTok}[1]{\textcolor[rgb]{0.31,0.60,0.02}{#1}}
\newcommand{\VariableTok}[1]{\textcolor[rgb]{0.00,0.00,0.00}{#1}}
\newcommand{\VerbatimStringTok}[1]{\textcolor[rgb]{0.31,0.60,0.02}{#1}}
\newcommand{\WarningTok}[1]{\textcolor[rgb]{0.56,0.35,0.01}{\textbf{\textit{#1}}}}
\usepackage{graphicx}
\makeatletter
\def\maxwidth{\ifdim\Gin@nat@width>\linewidth\linewidth\else\Gin@nat@width\fi}
\def\maxheight{\ifdim\Gin@nat@height>\textheight\textheight\else\Gin@nat@height\fi}
\makeatother
% Scale images if necessary, so that they will not overflow the page
% margins by default, and it is still possible to overwrite the defaults
% using explicit options in \includegraphics[width, height, ...]{}
\setkeys{Gin}{width=\maxwidth,height=\maxheight,keepaspectratio}
% Set default figure placement to htbp
\makeatletter
\def\fps@figure{htbp}
\makeatother
\setlength{\emergencystretch}{3em} % prevent overfull lines
\providecommand{\tightlist}{%
  \setlength{\itemsep}{0pt}\setlength{\parskip}{0pt}}
\setcounter{secnumdepth}{-\maxdimen} % remove section numbering
\ifLuaTeX
  \usepackage{selnolig}  % disable illegal ligatures
\fi

\begin{document}

\#Sample Data Analysis Project (Cycling Ridership Analysis)

In this project, we will import data collected by a bikeshare company in
NYC. The business task at hand is to examine the relationship/difference
between casual riders and members

\#\#Data Wrangling

Loading the required packages and importing data into Rstudio

\begin{Shaded}
\begin{Highlighting}[]
\FunctionTok{library}\NormalTok{(tidyverse)}
\FunctionTok{library}\NormalTok{(lubridate)}
\FunctionTok{library}\NormalTok{(readr)}
\end{Highlighting}
\end{Shaded}

\begin{Shaded}
\begin{Highlighting}[]
\NormalTok{january\_data }\OtherTok{\textless{}{-}} \FunctionTok{read\_csv}\NormalTok{(}\StringTok{"C:/Users/Stai Ndirangu/Desktop/Divvy Data/2021 CSV Data/202101{-}divvy{-}tripdata.csv"}\NormalTok{)}
\NormalTok{february\_data }\OtherTok{\textless{}{-}} \FunctionTok{read\_csv}\NormalTok{(}\StringTok{"C:/Users/Stai Ndirangu/Desktop/Divvy Data/2021 CSV Data/202102{-}divvy{-}tripdata.csv"}\NormalTok{)}
\NormalTok{march\_data }\OtherTok{\textless{}{-}} \FunctionTok{read\_csv}\NormalTok{(}\StringTok{"C:/Users/Stai Ndirangu/Desktop/Divvy Data/2021 CSV Data/202103{-}divvy{-}tripdata.csv"}\NormalTok{)}
\NormalTok{april\_data }\OtherTok{\textless{}{-}} \FunctionTok{read\_csv}\NormalTok{(}\StringTok{"C:/Users/Stai Ndirangu/Desktop/Divvy Data/2021 CSV Data/202104{-}divvy{-}tripdata.csv"}\NormalTok{)}
\NormalTok{may\_data }\OtherTok{\textless{}{-}} \FunctionTok{read\_csv}\NormalTok{(}\StringTok{"C:/Users/Stai Ndirangu/Desktop/Divvy Data/2021 CSV Data/202105{-}divvy{-}tripdata.csv"}\NormalTok{)}
\NormalTok{june\_data }\OtherTok{\textless{}{-}} \FunctionTok{read\_csv}\NormalTok{(}\StringTok{"C:/Users/Stai Ndirangu/Desktop/Divvy Data/2021 CSV Data/202106{-}divvy{-}tripdata.csv"}\NormalTok{)}
\NormalTok{july\_data }\OtherTok{\textless{}{-}} \FunctionTok{read\_csv}\NormalTok{(}\StringTok{"C:/Users/Stai Ndirangu/Desktop/Divvy Data/2021 CSV Data/202107{-}divvy{-}tripdata.csv"}\NormalTok{)}
\NormalTok{august\_data }\OtherTok{\textless{}{-}} \FunctionTok{read\_csv}\NormalTok{(}\StringTok{"C:/Users/Stai Ndirangu/Desktop/Divvy Data/2021 CSV Data/202108{-}divvy{-}tripdata.csv"}\NormalTok{)}
\NormalTok{september\_data }\OtherTok{\textless{}{-}} \FunctionTok{read\_csv}\NormalTok{(}\StringTok{"C:/Users/Stai Ndirangu/Desktop/Divvy Data/2021 CSV Data/202109{-}divvy{-}tripdata.csv"}\NormalTok{)}
\NormalTok{october\_data }\OtherTok{\textless{}{-}} \FunctionTok{read\_csv}\NormalTok{(}\StringTok{"C:/Users/Stai Ndirangu/Desktop/Divvy Data/2021 CSV Data/202110{-}divvy{-}tripdata.csv"}\NormalTok{)}
\NormalTok{november\_data }\OtherTok{\textless{}{-}} \FunctionTok{read\_csv}\NormalTok{(}\StringTok{"C:/Users/Stai Ndirangu/Desktop/Divvy Data/2021 CSV Data/202111{-}divvy{-}tripdata.csv"}\NormalTok{)}
\NormalTok{december\_data }\OtherTok{\textless{}{-}} \FunctionTok{read\_csv}\NormalTok{(}\StringTok{"C:/Users/Stai Ndirangu/Desktop/Divvy Data/2021 CSV Data/202112{-}divvy{-}tripdata.csv"}\NormalTok{)}
\end{Highlighting}
\end{Shaded}

Next we combine the imported dataframes into one big dataframe

\begin{Shaded}
\begin{Highlighting}[]
\NormalTok{whole\_year\_data }\OtherTok{\textless{}{-}} \FunctionTok{bind\_rows}\NormalTok{(january\_data ,february\_data ,march\_data ,april\_data ,may\_data ,june\_data ,july\_data,august\_data ,september\_data ,october\_data ,november\_data ,december\_data)}
\end{Highlighting}
\end{Shaded}

Now we separate the `started\_at' column into year, month, day of week

\begin{Shaded}
\begin{Highlighting}[]
\NormalTok{whole\_year\_data}\SpecialCharTok{$}\NormalTok{date }\OtherTok{\textless{}{-}} \FunctionTok{as.Date}\NormalTok{(whole\_year\_data}\SpecialCharTok{$}\NormalTok{started\_at) }
\NormalTok{whole\_year\_data}\SpecialCharTok{$}\NormalTok{month }\OtherTok{\textless{}{-}} \FunctionTok{format}\NormalTok{(}\FunctionTok{as.Date}\NormalTok{(whole\_year\_data}\SpecialCharTok{$}\NormalTok{date), }\StringTok{"\%m"}\NormalTok{)}
\NormalTok{whole\_year\_data}\SpecialCharTok{$}\NormalTok{day }\OtherTok{\textless{}{-}} \FunctionTok{format}\NormalTok{(}\FunctionTok{as.Date}\NormalTok{(whole\_year\_data}\SpecialCharTok{$}\NormalTok{date), }\StringTok{"\%d"}\NormalTok{)}
\NormalTok{whole\_year\_data}\SpecialCharTok{$}\NormalTok{year }\OtherTok{\textless{}{-}} \FunctionTok{format}\NormalTok{(}\FunctionTok{as.Date}\NormalTok{(whole\_year\_data}\SpecialCharTok{$}\NormalTok{date), }\StringTok{"\%Y"}\NormalTok{)}
\NormalTok{whole\_year\_data}\SpecialCharTok{$}\NormalTok{day\_of\_week }\OtherTok{\textless{}{-}} \FunctionTok{format}\NormalTok{(}\FunctionTok{as.Date}\NormalTok{(whole\_year\_data}\SpecialCharTok{$}\NormalTok{date), }\StringTok{"\%A"}\NormalTok{)}
\end{Highlighting}
\end{Shaded}

Then we add a column to include ride length

\begin{Shaded}
\begin{Highlighting}[]
\NormalTok{whole\_year\_data}\SpecialCharTok{$}\NormalTok{ride\_length }\OtherTok{\textless{}{-}}\FunctionTok{difftime}\NormalTok{(whole\_year\_data}\SpecialCharTok{$}\NormalTok{ended\_at,whole\_year\_data}\SpecialCharTok{$}\NormalTok{started\_at)}
\end{Highlighting}
\end{Shaded}

Next step is to check data type of ride length column and converting it
from factor to numeric so that we can run calculations on the data

\begin{Shaded}
\begin{Highlighting}[]
\FunctionTok{is.factor}\NormalTok{(whole\_year\_data}\SpecialCharTok{$}\NormalTok{ride\_length)}
\NormalTok{whole\_year\_data}\SpecialCharTok{$}\NormalTok{ride\_length }\OtherTok{\textless{}{-}} \FunctionTok{as.numeric}\NormalTok{(}\FunctionTok{as.character}\NormalTok{(whole\_year\_data}\SpecialCharTok{$}\NormalTok{ride\_length))}
\FunctionTok{is.numeric}\NormalTok{(whole\_year\_data}\SpecialCharTok{$}\NormalTok{ride\_length)}
\end{Highlighting}
\end{Shaded}

Final step in cleaning the data is checking for and removing bad data
i.e negative trips and testing trips

\begin{Shaded}
\begin{Highlighting}[]
\NormalTok{negative\_ride\_length }\OtherTok{\textless{}{-}}\NormalTok{ whole\_year\_data }\SpecialCharTok{\%\textgreater{}\%} \FunctionTok{count}\NormalTok{(ride\_length}\SpecialCharTok{\textless{}}\DecValTok{0}\NormalTok{)}
\NormalTok{whole\_year\_data\_v2 }\OtherTok{\textless{}{-}}\NormalTok{ whole\_year\_data[}\SpecialCharTok{!}\NormalTok{(whole\_year\_data}\SpecialCharTok{$}\NormalTok{ride\_length}\SpecialCharTok{\textless{}}\DecValTok{0}\NormalTok{),]}
\end{Highlighting}
\end{Shaded}

At this stage we can investigate our data and start gaining some
insights. Lets do some analysis on the clean data frame on ride length
variable and compare the descriptive stats of casuals and members

\begin{Shaded}
\begin{Highlighting}[]
\FunctionTok{summary}\NormalTok{(whole\_year\_data\_v2}\SpecialCharTok{$}\NormalTok{ride\_length)}
\FunctionTok{aggregate}\NormalTok{(whole\_year\_data\_v2}\SpecialCharTok{$}\NormalTok{ride\_length }\SpecialCharTok{\textasciitilde{}}\NormalTok{ whole\_year\_data\_v2}\SpecialCharTok{$}\NormalTok{member\_casual, }\AttributeTok{FUN =}\NormalTok{ mean)}
\FunctionTok{aggregate}\NormalTok{(whole\_year\_data\_v2}\SpecialCharTok{$}\NormalTok{ride\_length }\SpecialCharTok{\textasciitilde{}}\NormalTok{ whole\_year\_data\_v2}\SpecialCharTok{$}\NormalTok{member\_casual, }\AttributeTok{FUN =}\NormalTok{ median)}
\FunctionTok{aggregate}\NormalTok{(whole\_year\_data\_v2}\SpecialCharTok{$}\NormalTok{ride\_length }\SpecialCharTok{\textasciitilde{}}\NormalTok{ whole\_year\_data\_v2}\SpecialCharTok{$}\NormalTok{member\_casual, }\AttributeTok{FUN =}\NormalTok{ max)}
\FunctionTok{aggregate}\NormalTok{(whole\_year\_data\_v2}\SpecialCharTok{$}\NormalTok{ride\_length }\SpecialCharTok{\textasciitilde{}}\NormalTok{ whole\_year\_data\_v2}\SpecialCharTok{$}\NormalTok{member\_casual, }\AttributeTok{FUN =}\NormalTok{ min)}
\end{Highlighting}
\end{Shaded}

We can go further by computing average ride time by day for members vs
casual users

\begin{Shaded}
\begin{Highlighting}[]
\FunctionTok{aggregate}\NormalTok{(whole\_year\_data\_v2}\SpecialCharTok{$}\NormalTok{ride\_length }\SpecialCharTok{\textasciitilde{}}\NormalTok{ whole\_year\_data\_v2}\SpecialCharTok{$}\NormalTok{member\_casual }\SpecialCharTok{+}\NormalTok{ whole\_year\_data\_v2}\SpecialCharTok{$}\NormalTok{day\_of\_week, }\AttributeTok{FUN =}\NormalTok{ mean)}
\end{Highlighting}
\end{Shaded}

The order of the days of the week in the values generated above is wrong
so lets correct that and run the fuction again

\begin{Shaded}
\begin{Highlighting}[]
\NormalTok{whole\_year\_data\_v2}\SpecialCharTok{$}\NormalTok{day\_of\_week }\OtherTok{\textless{}{-}} \FunctionTok{ordered}\NormalTok{(whole\_year\_data\_v2}\SpecialCharTok{$}\NormalTok{day\_of\_week, }\AttributeTok{levels=}\FunctionTok{c}\NormalTok{(}\StringTok{"Sunday"}\NormalTok{, }\StringTok{"Monday"}\NormalTok{, }\StringTok{"Tuesday"}\NormalTok{, }\StringTok{"Wednesday"}\NormalTok{, }\StringTok{"Thursday"}\NormalTok{, }\StringTok{"Friday"}\NormalTok{, }\StringTok{"Saturday"}\NormalTok{))}
\end{Highlighting}
\end{Shaded}

\begin{Shaded}
\begin{Highlighting}[]
\NormalTok{counts }\OtherTok{\textless{}{-}} \FunctionTok{aggregate}\NormalTok{(whole\_year\_data\_v2}\SpecialCharTok{$}\NormalTok{ride\_length }\SpecialCharTok{\textasciitilde{}}\NormalTok{ whole\_year\_data\_v2}\SpecialCharTok{$}\NormalTok{member\_casual }\SpecialCharTok{+}\NormalTok{ whole\_year\_data\_v2}\SpecialCharTok{$}\NormalTok{day\_of\_week, }\AttributeTok{FUN =}\NormalTok{ mean)}
\end{Highlighting}
\end{Shaded}

Since our data is now well cleaned and ready for analysis, lets analyze
ridership data by type and weekday

\begin{Shaded}
\begin{Highlighting}[]
\NormalTok{analysis\_results }\OtherTok{\textless{}{-}}\NormalTok{ whole\_year\_data\_v2 }\SpecialCharTok{\%\textgreater{}\%} 
  \FunctionTok{mutate}\NormalTok{(}\AttributeTok{weekday=}\FunctionTok{wday}\NormalTok{(started\_at, }\AttributeTok{label =} \ConstantTok{TRUE}\NormalTok{)) }\SpecialCharTok{\%\textgreater{}\%} 
  \FunctionTok{group\_by}\NormalTok{(member\_casual, weekday) }\SpecialCharTok{\%\textgreater{}\%} 
  \FunctionTok{summarise}\NormalTok{(}\AttributeTok{number\_of\_rides=}\FunctionTok{n}\NormalTok{(), }\AttributeTok{average\_duration =} \FunctionTok{mean}\NormalTok{(ride\_length)) }\SpecialCharTok{\%\textgreater{}\%} 
  \FunctionTok{arrange}\NormalTok{(member\_casual, weekday)}
\end{Highlighting}
\end{Shaded}

\#\#Visualize ridership data by member type

With our data now properly sorted and a final dataframe generated for
final analysis, lets first load ggplot2 and then create a visual for
ridership data by member type

\begin{Shaded}
\begin{Highlighting}[]
\FunctionTok{library}\NormalTok{(ggplot2)}

\FunctionTok{ggplot}\NormalTok{(}\AttributeTok{data =}\NormalTok{ analysis\_results, }\FunctionTok{aes}\NormalTok{(}\AttributeTok{x =}\NormalTok{ weekday, }\AttributeTok{y =}\NormalTok{ number\_of\_rides, }\AttributeTok{fill =}\NormalTok{ member\_casual)) }\SpecialCharTok{+}
  \FunctionTok{geom\_col}\NormalTok{(}\AttributeTok{position =} \StringTok{"dodge"}\NormalTok{)}
\end{Highlighting}
\end{Shaded}

We can also visualize ridership data by average duration

\begin{Shaded}
\begin{Highlighting}[]
\FunctionTok{ggplot}\NormalTok{(}\AttributeTok{data =}\NormalTok{ analysis\_results, }\FunctionTok{aes}\NormalTok{(}\AttributeTok{x =}\NormalTok{ weekday, }\AttributeTok{y =}\NormalTok{ average\_duration, }\AttributeTok{fill =}\NormalTok{ member\_casual))}\SpecialCharTok{+}
  \FunctionTok{geom\_col}\NormalTok{(}\AttributeTok{position =} \StringTok{"dodge"}\NormalTok{)}
\end{Highlighting}
\end{Shaded}

\#\#Exporting the csv files for further analysis

More visualization and creation of reports will be done later and on
other tools in this case Power Bi or Tableau. To do that, we copy the
dataframes developed here into csv files and save them in our local
computer

\begin{Shaded}
\begin{Highlighting}[]
\FunctionTok{write.csv}\NormalTok{(analysis\_results, }\AttributeTok{file =} \StringTok{"C:}\SpecialCharTok{\textbackslash{}\textbackslash{}}\StringTok{Users}\SpecialCharTok{\textbackslash{}\textbackslash{}}\StringTok{Stai Ndirangu}\SpecialCharTok{\textbackslash{}\textbackslash{}}\StringTok{Desktop}\SpecialCharTok{\textbackslash{}\textbackslash{}}\StringTok{Divvy Data}\SpecialCharTok{\textbackslash{}\textbackslash{}}\StringTok{Analysis results}\SpecialCharTok{\textbackslash{}\textbackslash{}}\StringTok{analysis\_results.csv"}\NormalTok{)}
\FunctionTok{write.csv}\NormalTok{(counts, }\AttributeTok{file =} \StringTok{"C:}\SpecialCharTok{\textbackslash{}\textbackslash{}}\StringTok{Users}\SpecialCharTok{\textbackslash{}\textbackslash{}}\StringTok{Stai Ndirangu}\SpecialCharTok{\textbackslash{}\textbackslash{}}\StringTok{Desktop}\SpecialCharTok{\textbackslash{}\textbackslash{}}\StringTok{Divvy Data}\SpecialCharTok{\textbackslash{}\textbackslash{}}\StringTok{Analysis results}\SpecialCharTok{\textbackslash{}\textbackslash{}}\StringTok{counts.csv"}\NormalTok{)}
\end{Highlighting}
\end{Shaded}


\end{document}
